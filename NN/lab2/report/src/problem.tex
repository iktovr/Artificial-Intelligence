\section*{Лабораторная работа \textnumero 2}

{\bfseries Тема:} Линейная нейронная сеть. Правило обучения Уидроу-Хоффа.

{\bfseries Цель работы:} Исследование свойств линейной нейронной сети и алгоритмов её обучения, применения сети в задачах аппроксимации и фильтрации.

{\bfseries Основные этапы работы:}
\begin{enumerate}
	\item Использовать линейную нейронную сеть с задержками для аппроксимации функции. В качестве метода обучения использовать адаптацию.
	\item Использовать линейную нейронную сеть в качестве адаптивного фильтра для подавления помех. Для настройки весовых коэффициентов использовать метод наименьших квадратов.
\end{enumerate}

{\bfseries Вариант:} 9

\begin{align*}
x &= \sin(t^2 - 2t + 3),\ t \in [0, 6],\ h = 0.025 \\
y &= \sin(t^2 - 2t),\ t \in [0, 6],\ h = 0.025 \\
x &= \sin(t^2 - 2t + 3)
\end{align*}
\pagebreak
