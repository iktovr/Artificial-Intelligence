\section*{Лабораторная работа \textnumero 1}

{\bfseries Тема:} Персептроны. Процедура обучения Розенблатта.

{\bfseries Цель работы:} Исследование свойств персептрона Розенблатта и его применение для решения задачи распознавания образов.

{\bfseries Основные этапы работы:}
\begin{enumerate}
	\item Для первой обучающей выборки построить и обучить сеть, которая будет правильно относить точки к двум классам. Отобразить дискриминантную линию и проверить качество обучения.
	\item Изменить обучающее множество так, чтобы классы стали линейно неразделимыми. Проверить возможности обучения по правилу Розенблатта.
	\item Для второй обучающей выборки построить и обучить сеть, которая будет правильно относить точки к четырем классам. Отобразить дискриминантную линию и проверить качество обучения.
\end{enumerate}

{\bfseries Вариант:} 9

\begin{center}
\begin{tabular}{|c|c|}
\hline
Data & Labels \\
\hline
$\begin{bmatrix}
-1.1 & 1.8 & 4.8 & 1.2 & -1.2 & 2.5 \\
-4.3 & -1 & -1 & -3.5 & -3.4 & 3.7
\end{bmatrix}$
&
$\begin{bmatrix}
0 & 1 & 1 & 1 & 0 & 1
\end{bmatrix}$ \\ [4ex]
$\begin{bmatrix}
4.6 & -1 & -0.3 & -1.1 & 0.5 & 4.9 & 0.3 & -3.9 \\
1.7 & 4.3 & -2.7 & 2 & 2.5 & 4.6 & 4.6 & -4.5
\end{bmatrix}$
&
$\begin{bmatrix}
1 & 0 & 1 & 1 & 1 & 1 & 0 & 1 \\
1 & 0 & 1 & 0 & 0 & 0 & 0 & 1
\end{bmatrix}$ \\
\hline
\end{tabular}
\end{center}
\pagebreak
