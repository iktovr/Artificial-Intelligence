\section*{Лабораторная работа \textnumero 3}

{\bfseries Тема:} Многослойные сети. Алгоритм обратного распространения ошибки.

{\bfseries Цель работы:} Исследование свойств многослойной нейронной сети прямого распространения и алгоритмов ее обучения, применение сети в задачах классификации и аппроксимации функции.

{\bfseries Основные этапы работы:}
\begin{enumerate}
	\item Использовать многослойную нейронную сеть для классификации точек в случае, когда классы не являются линейно разделимыми.
	\item Использовать многослойную нейронную сеть для аппроксимации функции.
\end{enumerate}

{\bfseries Вариант:} 9

{\bfseries Алгебраические линии:}
\begin{itemize}
	\item Эллипс: $a = 0.2,\ b = 0.2,\ \alpha = 0,\ x_0 = -0.2,\ y_0 = 0$
	\item Эллипс: $a = 0.7,\ b = 0.5,\ \alpha = -\pi/3,\ x_0 = 0,\ y_0 = 0$
	\item Эллипс: $a = 1,\ b = 1,\ \alpha = 0,\ x_0 = 0,\ y_0 = 0$
\end{itemize}

{\bfseries Функция:} $x = \sin(t ^ 2 - 2t + 5)$

\pagebreak
