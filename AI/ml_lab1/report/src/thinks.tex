\section{Выводы}

В ходе выполнения лабораторной работы я реализовал некоторые алгоритмы машинного обучения и испытал их на данных, обработанных в предыдущей работе. Лучше всего себя показал метод k ближайших соседей, вероятно из-за большого количетства данных, затем --- метод опорных векторов и наивный байесовский классификатор, наконец --- логистическая регрессия. Я сравнил результаты с готовыми версиями соответствующих алгоритмов, но при тех же параметрах они показали себя не лучше. Результаты работы подтверждают предыдущие предположения --- данная задача решаема методами машинного обучения, причем с очень хорошей точностью 93-99\%.
\pagebreak
